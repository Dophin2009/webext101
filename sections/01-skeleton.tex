% -----------------------------------------------
% chktex-file 44
\documentclass[../index.tex]{subfiles}

% -----------------------------------------------

\begin{document}

% -----------------------------------------------

\renewcommand{\sectiontitle}{Skeleton of an extension}
\section{\sectiontitle}

% ---------------------------
\renewcommand{\currenttitle}{File structure}
\begin{frame}[fragile]{\currenttitle}
  An basic extension typically looks something like this:

  \begin{forest}
      for tree={
        font=\ttfamily\footnotesize,
        grow'=0,
        child anchor=west,
        parent anchor=south,
        anchor=west,
        calign=first,
        edge path={
          \noexpand\path [draw, \forestoption{edge}]
          (!u.south west) +(7.5pt,0) |- node[fill,inner sep=1.25pt] {} (.child anchor)\forestoption{edge label};
        },
        before typesetting nodes={
          if n=1
            {insert before={[,phantom]}}
            {}
        },
        fit=band,
        before computing xy={l=15pt},
      }
    [root
      [fg
        [content-script-1.js]
        [content-script-2.js]
      ]
      [bg
        [script-1.js]
        [script-2.js]
      ]
      [manifest.json]
    ]
  \end{forest}
\end{frame}

% ---------------------------
\renewcommand{\currenttitle}{The manifest}
\begin{frame}[fragile]{\currenttitle}
  Every extension must have a \texttt{manifest.json} file that tells the
  browser what assets to load and other settings:

  \begin{lstlisting}[language=json]
{
  "manifest_version": 2,
  "name": "Extension name",
  "version": "1.0",
  "description": "Description of the extension's functionality",
  "content_scripts": [
    {
      "matches": ["://*.mozilla.org/*"],
      "js": ["fg/index.js"]
    }
  ],
  "background": {
    "scripts": ["bg/index.js", "lib/jquery.js"]
  }
}
  \end{lstlisting}
\end{frame}

% ---------------------------
\renewcommand{\currenttitle}{Let's create a manifest}
\begin{frame}[fragile]{\currenttitle}
  Create the file \texttt{/manifest.json} and add:
  \lstinputlisting{\emojisubdir/manifest.json}
\end{frame}

% ---------------------------
\renewcommand{\currenttitle}{Let's create a manifest: breaking it down}
\begin{frame}[fragile]{\currenttitle}
  Some standard metadata:
  \lstinputlisting[firstline=3,lastline=5]{\emojisubdir/manifest.json}

  Set the extension icon:
  \lstinputlisting[firstline=6,lastline=9]{\emojisubdir/manifest.json}
\end{frame}

\begin{frame}[fragile]{\currenttitle}
  Define the content scripts:
  \lstinputlisting[firstline=10,lastline=15]{\emojisubdir/manifest.json}

  We'll start with using \texttt{index.js}, but \texttt{emojimap.js} will be
  useful later.
\end{frame}

% ---------------------------
\renewcommand{\currenttitle}{Content scripts}
\begin{frame}[fragile]{\currenttitle}
  \textbf{Content scripts} are \textbf{injected} into web pages and run in the
  context of a particular web page

  They\dots
  \begin{itemize}
    \item are loaded when the page is loaded
    \item \textbf{can} read and modify the contents of the webpage
    \item \textbf{can} use a small subset of WebExtension APIs
    \item \textbf{cannot} access browser storage and other APIs
    \item \textbf{are blocked} on certain pages as dictated by the browser
    \item \textbf{must} use the messaging API to communicate with background scripts
  \end{itemize}
\end{frame}

% ---------------------------
\renewcommand{\currenttitle}{Hello, World! in content scripts}
\begin{frame}[fragile]{\currenttitle}
  Create the folder \texttt{/js} and the file \texttt{/js/emojimap.js} (for later use) \\[1.5em]

  Create the file \texttt{/js/index.js} with a standard Hello, World! program:
  \begin{lstlisting}[language=ES6,basicstyle=\ttfamily\small]
    console.log("Hello, world!");
  \end{lstlisting}
\end{frame}

% ---------------------------
\renewcommand{\currenttitle}{Let's test it!}
\begin{frame}[fragile]{\currenttitle}
  \textbf{Firefox (the better browser)}:
  \begin{enumerate}
    \item Navigate to \texttt{about:debugging} in the address bar
    \item Enter the "This Firefox" tab
    \item Click "Load Temporary Add-on..."
    \item Select the \texttt{emojisub} folder
  \end{enumerate}

  \textbf{Chrome}:
  \begin{enumerate}
    \item Three dots ($\vdots$) \textrightarrow "More tools" \textrightarrow
          "Extensions"
    \item Enable "Developer mode"
    \item Click "Load unpacked"
    \item Select the \texttt{emojisub} folder
  \end{enumerate}
\end{frame}

% ---------------------------
\begin{frame}[fragile]{\currenttitle}
  Open a page (e.g. \texttt{duckduckgo.com}) and see if the console prints
  "Hello, world!"
\end{frame}

% ---------------------------
\renewcommand{\currenttitle}{Implementing the actual extension}
\begin{frame}[fragile]{\currenttitle}
  For simplicity's sake, just copy the code into \texttt{index.js} and
  \texttt{emojimap.js}:

  \begin{itemize}
    \item \href{https://github.com/Dophin2009/webext101/blob/master/emojisub/js/emojimap.js}{\texttt{/js/emojimap.js}}
    \item \href{https://github.com/Dophin2009/webext101/blob/master/emojisub/js/index.js}{\texttt{/js/index.js}}
  \end{itemize}
\end{frame}

% ---------------------------
\begin{frame}[fragile]{\currenttitle}
  Notice a few things from a quick perusal of the code:

  \begin{itemize}
    \item Variables are shared across scripts (as with plain JS)
    \item Content scripts can modify the contents of the web page
    \item A \texttt{MutationObserver} is useful for applying changes across the
          page at all times
    \item We can use rather modern JS because the browser supports it (also a
          gotcha)
  \end{itemize}
\end{frame}

% ---------------------------
\renewcommand{\currenttitle}{Let's retest}
\begin{frame}{\currenttitle}
  \textbf{Firefox}:
  \begin{enumerate}
    \item Click "Reload" in the bottom right corner of the extension card
  \end{enumerate}

  \textbf{Chrome}:
  \begin{enumerate}
    \item Click the reload icon in the bottom right corner of the extension
          card
  \end{enumerate}
\end{frame}

% ---------------------------
\renewcommand{\currenttitle}{Background scripts}
\begin{frame}{\currenttitle}
  Background scripts are loaded as soon as the extension is loaded and can be
  used to maintain persistent state. \\[1em]

  They\dots
  \begin{itemize}
    \item \textbf{can} use any of the WebExtension APIs (as long as permissions
          are granted)
    \item \textbf{cannot} access the content of web pages
  \end{itemize}

  Mostly out of scope for today.
\end{frame}

% ---------------------------
\renewcommand{\currenttitle}{Other assets}
\begin{frame}[fragile]{\currenttitle}
  We can include other assets (images, stylesheets, etc.) in our extension and
  load them from content scripts or page scripts.

  In \texttt{manifest.json}:
  \begin{lstlisting}[language=json]
    "web_accessible_resources": ["images/my-image.png"]
  \end{lstlisting}

  In a script:
  \begin{lstlisting}[language=ES6]
    browser.runtime.getURL("images/my-image.png")
  \end{lstlisting}

  See \href{https://developer.mozilla.org/en-US/docs/Mozilla/Add-ons/WebExtensions/manifest.json/web_accessible_resources}{MDN docs}.
\end{frame}

% -----------------------------------------------

\end{document}
