% -----------------------------------------------
% chktex-file 44
\documentclass[../index.tex]{subfiles}

% -----------------------------------------------

\begin{document}

% -----------------------------------------------

\renewcommand{\sectiontitle}{More advanced tooling}
\section{\sectiontitle}

% ---------------------------
\renewcommand{\currenttitle}{Using external libraries}
\begin{frame}[fragile]{\currenttitle}
  You may want to pull in external libraries into your extension. \\[1em]

  In the JS world, you have many (too many) options:
  \begin{itemize}
    \item Store a copy of the library in your extension
    \item Use a CDN
    \item Use a package manager to download the library
    \item Use a package manager + bundler
  \end{itemize}
\end{frame}

% ---------------------------
\renewcommand{\currenttitle}{Storing a copy of the library}
\begin{frame}[fragile]{\currenttitle}
  Let's say we want to use the jQuery library:

  \begin{enumerate}
    \item Download\footnote{https://code.jquery.com/jquery-3.6.0.min.js} into
          your project
    \item Load the file in the necessary places
  \end{enumerate}
\end{frame}

% ---------------------------
\begin{frame}[fragile]{\currenttitle}
  To use in content scripts, we just have to add it to our
  \texttt{manifest.json}:

  \begin{lstlisting}[language=ES6]
{
  ...
  "content_scripts": [
    {
      "matches": [...],
      "js": ["jquery-3.6.0.min.js", ...]
    }
  ],
  ...
}
  \end{lstlisting}

  The jQuery global (\texttt{\$}) is now available in all content scripts.
\end{frame}

% ---------------------------
\begin{frame}[fragile]{\currenttitle}
  To use in a background scripts, we can do the same thing:

  \begin{lstlisting}[language=ES6]
{
  ...
  "background": [
    "scripts": ["jquery-3.6.0.min.js", ...]
  ],
  ...
}
  \end{lstlisting}
\end{frame}

% ---------------------------
\renewcommand{\currenttitle}{Using a package manager}
\begin{frame}[fragile]{\currenttitle}
  We can use the Node Package Manager (npm) that comes with Node.js to install
  jQuery:

  \begin{lstlisting}[language=Bash,basicstyle=\ttfamily\small]
    npm init            # initialize package.json
    npm install jquery  # add jquery as a dependency
  \end{lstlisting}

  \vspace*{1em}
  This will create a \texttt{node\_modules} in your project directory.

\end{frame}

% ---------------------------
\begin{frame}[fragile]{\currenttitle}
  We can now load jQuery from inside \texttt{/node\_modules}:

  \begin{lstlisting}[language=json]
{
  ...
  "content_scripts": [
    {
      "matches": [...],
      "js": ["node_modules/jquery/dist/jquery.js", ...]
    }
  ]
  ...
}
  \end{lstlisting}
\end{frame}

% ---------------------------
\begin{frame}[fragile]{\currenttitle}
  The npm registry (\url{npmjs.com}) has \textit{millions} of
  packages/libraries available for use. \\[2em]

  \textbf{Beware: Using npm is no different from downloading random crap off
  the internet.}

  \textbf{The JS package ecosystem is ripe with security issues.}
\end{frame}

% ---------------------------
\renewcommand{\currenttitle}{Using a package manager + bundler}
\begin{frame}[fragile]{\currenttitle}
  Modern web practices typically involve using a \textbf{bundler} such as
  Webpack or Parcel to combine all your assets into one or multiple JS files.
  \\[1em]

  This means that there will be an extra compilation step between writing code
  and testing your extension. \\[3em]

  Let me know if you're interested in exploring this.
\end{frame}

% ---------------------------
\renewcommand{\currenttitle}{Typescript}
\begin{frame}[fragile]{\currenttitle}
  You will have a much more software engineering experience with
  \textbf{Typescript}:

  \begin{itemize}
    \item Static type declarations and checking
      \begin{itemize}
        \item Eliminate a whole class of errors
        \item Better code completion
        \item Easier to test
        \item More readable
      \end{itemize}
      \item First-class new feature support
    \item Stricter lints
  \end{itemize}
\end{frame}

% -----------------------------------------------

\end{document}
