% -----------------------------------------------
% chktex-file 44
\documentclass[../index.tex]{subfiles}

% -----------------------------------------------

\begin{document}

% -----------------------------------------------

\renewcommand{\sectiontitle}{Web development is a gotcha minefield}
\section{\sectiontitle}

% ---------------------------
\renewcommand{\currenttitle}{Gotcha \#1: Browser incompatibilities}
\begin{frame}[fragile]{\currenttitle}
  Different browsers often support different sets of APIs, not just for
  WebExtensions: \\[2em]

  See:
  \begin{itemize}
    \item Can I Use for JS, HTML, CSS, etc. support across browsers
          \footnote{\scriptsize \url{https://caniuse.com/}}
    \item WebExtensions API support in various browsers
          \footnote{\scriptsize \url{https://developer.mozilla.org/en-US/docs/Mozilla/Add-ons/WebExtensions/Browser_support_for_JavaScript_APIs}}
  \end{itemize}

\end{frame}

% ---------------------------
\begin{frame}[fragile]{\currenttitle}
  Remember namespaces:
  \begin{itemize}
    \item Firefox \textendash{} \texttt{browser} namspace ;
          \texttt{Promise}-based (modern and maintainable)
    \item Chrome and co. \textendash{} \texttt{chrome} namspace ;
          callback-based (old and error-prone)
    \item Edge \textendash{} \texttt{browser} namespace ;
          callback-based (Edge sucks)
  \end{itemize}
\end{frame}

% ---------------------------
\begin{frame}[fragile]{\currenttitle}
  \begin{lstlisting}[language=ES6]
// In Chrome
function logCookie(c) {
  if (chrome.runtime.lastError) {
    console.error(chrome.runtime.lastError);
  } else {
    console.log(c);
  }
}

// We pass the function `logCookie` as a callback into 
// `cookies.set` to deal with the return value.
chrome.cookies.set(
  {url: "https://developer.mozilla.org/"},
  logCookie
);
  \end{lstlisting}
\end{frame}

% ---------------------------
\begin{frame}[fragile]{\currenttitle}
  \begin{lstlisting}[language=ES6]
// The same code in Firefox
function logCookie(c) {
  console.log(c);
}

function logError(e) {
  console.error(e);
}

// `cookies.set` returns a `Promise`, and we call the
// `then` method to deal with the value.
let setCookie = browser.cookies.set(
  {url: "https://developer.mozilla.org/"}
);
setCookie.then(logCookie, logError);
  \end{lstlisting}
\end{frame}

% ---------------------------
\newcommand{\promisesfootnote}{\footnote{\url{https://developer.mozilla.org/en-US/docs/Web/JavaScript/Reference/Global_Objects/Promise}}}
\newcommand{\asyncfootnote}{\footnote{\url{https://developer.mozilla.org/en-US/docs/Learn/JavaScript/Asynchronous/Async_await}}}
\newcommand{\polyfillfootnote}{\footnote{\url{https://github.com/mozilla/webextension-polyfill}}}
\begin{frame}[fragile]{\currenttitle}
  \textbf{Solution}: Learn the differences and deal with them \\[1em]

  \textbf{Better solution}: Learn \texttt{Promise}s\promisesfootnote{}
  and \texttt{async}/\texttt{await}\asyncfootnote{}
  and use Mozilla's \texttt{webextension-polyfill}\polyfillfootnote{}
  to use \texttt{browser} and \texttt{Promise}s everywhere
\end{frame}

% ---------------------------
\renewcommand{\currenttitle}{Gotcha \#2: Manifest keys}
\begin{frame}[fragile]{\currenttitle}
  Beware:
  \begin{itemize}
    \item Some browsers may not support certain keys\footnotemark{} (e.g. Edge does not
          support \texttt{commands} key)
          \footnotetext{\url{https://developer.mozilla.org/en-US/docs/Mozilla/Add-ons/WebExtensions/manifest.json#browser_compatibility}}
    \item Some browsers may require some settings in the \texttt{"browser\_specific\_settings"} field
  \end{itemize}

  \vspace*{1em}
  \textbf{Solution}: Reading Is Fundamental
\end{frame}

% -----------------------------------------------

\end{document}
